\documentclass[11pt]{article}

% Created by @btskinner (GitHub)

% -- NOTES -----------------------------

% Adjust this template as needed to create a CV that meets your
% specific needs. Add / delete / move the sections in the
% document body below to make a personalized CV. In general, sections
% that need user input versus those that should be left as is (except
% by more advanced users who want to fiddle with details of the
% template) are separated like this:
%
% - Commented sections using "====" are for user modification.
% - Commented sections using "----" do not need to changed.
%
% This template has been munged together from lots of little code
% snippets over the years, so I can't take complete credit. I've tried
% to attribute blocks of code I took from elsewhere.

% ==============================================================================
% USER SETTINGS: CHANGE THINGS HERE
% ==============================================================================

% ======================================
% PERSONAL INFO
% ======================================

% Change the value in the second set of {} to you information. If you
% want multiple lines (like for the address), use \\ to split lines.

\newcommand{\myname}{Wayne Yu Wang}     % your name
\newcommand{\websiteurl}{{https://ywa136.github.io}} % your site url
\newcommand{\websitename}{me.com}       % short url name for printing
\newcommand{\phone}{734.277.5499}       % your phone number
\newcommand{\email}{wayneyw@umich.edu}  % your email
\newcommand{\twitterurl}{{https://twitter.com/wayneywang}}  % twitter url
\newcommand{\twitterhandle}{@wayneywang}      % twitter handle
\newcommand{\address}{                  % NB: Assumes 3 lines
  Department of Statistics, University of Michigan \\
  323 West Hall \\
  1085 South University \\    
  Ann Arbor, MI 48109                   
}

% ======================================
% FILTER TO ITEMS AFTER THIS YEAR
% ======================================

% If you want to list publications / work products only from the past
% XX years, set this value to the 4-digit year that's one less than
% the first year you want.
%
% If only 2015 - present ---> set to 2014

\newcommand{\recentyear}{1900}  % artificially low year to include everything

% ------------------------------------------------------------------------------
% HEADER STUFF (CAN LEAVE ALONE UNLESS YOU WANT TO PERSONALIZE)
% ------------------------------------------------------------------------------

% -------------------------------
% Packages
% -------------------------------

\usepackage[margin=1in]{geometry} % for margins
\usepackage[american]{babel}      % for language
\usepackage{csquotes}             % idk, babel/biber like it
\usepackage[style=apa,
backend=biber,
sortcites=true,
sorting=ydmdnt,
language=american]{biblatex}      % for reference sections
\usepackage{hyperref}             % to add links
\usepackage{fancyhdr}             % for better headers/footers
\usepackage{titlesec}             % to adjust section headers
\usepackage{tabularx}             % for fluid tables
\usepackage{datetime}             % to make better date formate
\usepackage[inline]{enumitem}     % more control over lists

% -------------------------------
% Macros
% -------------------------------

\newcommand{\RR}{\raggedright\arraybackslash} % left justified
\newcommand{\RL}{\raggedleft\arraybackslash}  % right justified

% for tables to keep consistent alignment across sections
\newcolumntype{\twocols}{>{\RR}p{1.25in}>{\RR}X}
\newcolumntype{\threecols}{>{\RR}p{1.25in}>{\RR}X>{\RL}p{1in}}

% -------------------------------
% BibLaTeX sorting scheme
% -------------------------------

% Create a new sorting scheme that will sort in reverse chronological
% order, taking months into account correctly (i.e., in chronological
% order and not alphabetical order)

\DeclareSortingScheme{ydmdnt}{
  \sort{
    \field{presort}
  }
  \sort[final]{
    \field{sortkey}
  }
  \sort[direction=descending]{
    \field{sortyear}
    \field{year}
    \literal{9999}
  }
  \sort[direction=descending]{
    \field[padside=left,padwidth=2,padchar=0]{month}
    \literal{99}
  }
  \sort[direction=descending]{
    \field[padside=left,padwidth=2,padchar=0]{day}
    \literal{99}
  }
  \sort{
    \field{sortname}
    \field{author}
    \field{editor}
    \field{translator}
    \field{sorttitle}
    \field{title}
  }
  \sort{
    \field{sorttitle}
  }
  \sort[direction=descending]{
    \field[padside=left,padwidth=4,padchar=0]{volume}
    \literal{9999}
  }
}

% -------------------------------
% BibLaTeX options
% -------------------------------

% language mapping
\DeclareLanguageMapping{american}{american-apa}

% ignore addendum and note fields that are sometimes useful, but not here
\DeclareSourcemap{
  \maps[datatype=bibtex]{
    \map{
      \step[fieldset=addendum, null]
      \step[fieldset=note, null]
    } 
  }
}

% macro so titles are converted to doi, url, or isbn link (if available)
% h\t https://tex.stackexchange.com/a/48409
\ExecuteBibliographyOptions{doi=false,url=false,isbn=false}
\newbibmacro{string+doiurlisbn}[1]{%
  \iffieldundef{doi}{%
    \iffieldundef{url}{%
      \iffieldundef{isbn}{%
        \iffieldundef{issn}{%
          #1%
        }{%
          \href{http://books.google.com/books?vid=ISSN\thefield{issn}}{#1}%
        }%
      }{%
        \href{http://books.google.com/books?vid=ISBN\thefield{isbn}}{#1}%
      }%
    }{%
      \href{\thefield{url}}{#1}%
    }%
  }{%
    \href{http://dx.doi.org/\thefield{doi}}{#1}%
  }%
}

\DeclareFieldFormat{title}{\usebibmacro{string+doiurlisbn}{\mkbibemph{#1}}}
\DeclareFieldFormat[article,incollection]{title}%
{\usebibmacro{string+doiurlisbn}{\mkbibquote{#1}}}

% no letter for same year entries
\defbibenvironment{mybib}
{\list
  {}
  {\setlength{\leftmargin}{\bibhang}%
    \setlength{\itemindent}{-\leftmargin}%
    \setlength{\itemsep}{\bibitemsep}%
    \setlength{\parsep}{\bibparsep}}}
{\endlist}
{\clearfield{extradate}\item}

% filter by year (only print those after year value in \recentyear
% macro above)
\defbibcheck{recent}{
  \iffieldint{year}
  {\ifnumless{\thefield{year}}{\recentyear}
    {\skipentry}
    {}}
  {\skipentry}
}

% -------------------------------
% Document options
% -------------------------------

% keep lists tight (no extra whitespace)
\setlist{nolistsep}

% This section determines how the sections are displayed. Adjust these
% as you like using the titlesec package options.

% section
\titleformat{\section}[display]
{\Large\bfseries}
{\filleft\Large\sectiontitlename~\thesection}
{}
{\vspace{-1ex}}
[]

% subsections
\titleformat{\subsection}[display]
{\bfseries}
{\filleft\subsectiontitlename~\thesubsection}
{}
{}
[\vspace{-1ex}]

% subsubsections
\titleformat{\subsubsection}[display]
{}
{\filleft\subsubsectiontitlename~\thesubsubsection}
{}
{}
[\vspace{-1ex}]

% -------------------------------
% Headers and Footers
% -------------------------------

% All pages after the first will have your name and page number in the
% upper right corner; all pages will have the date of compilation (the
% latest update) in the bottom right corner. You can move these.

\fancypagestyle{first}{
   \fancyhead[R]{}
   \fancyhead[L]{}
   \fancyfoot[R]{{\itshape Updated: \today}}
   \fancyfoot[L]{}
   \fancyfoot[C]{}
   \renewcommand{\headrulewidth}{0.0pt}
   \renewcommand{\footrulewidth}{0.0pt}}

\fancypagestyle{rest}{
   \fancyhead[R]{{\itshape \myname\, $\mid$ \thepage}}
   \fancyhead[L]{}
   \fancyfoot[R]{{\itshape Updated: \today}}
   \fancyfoot[L]{}
   \fancyfoot[C]{}
   \renewcommand{\headrulewidth}{0.0pt}
   \renewcommand{\footrulewidth}{0.0pt}}

%%%%%%%%%%%%%%%%%%%%%%%%%%%%%%%%%%%%%%%%%%%%%%%%%%%%%%%%%%%%%%%%%%%%%%%%%%%%%%%%
% BEGIN DOCUMENT
%%%%%%%%%%%%%%%%%%%%%%%%%%%%%%%%%%%%%%%%%%%%%%%%%%%%%%%%%%%%%%%%%%%%%%%%%%%%%%%%

% ======================================
% Bibliography
% ======================================

% path to your CV bibliography (.bib) file; should be same name as
% your *.tex file if you want to use the makefile
\addbibresource{./cv.bib}
 
\pagestyle{rest}
\begin{document}
\thispagestyle{first}

% -------------------------------
% Front page header
% -------------------------------

% Adjust this as necessary if you need to add or delete items from the
% top. You can also create your own. The structure right now is:
%
% |----------------------------------------------|
% |       Centered box across page for Name      |
% |----------------------------------------------|
% | Left-align 1/2 box    | Right-align 1/2 box  |
% |                       |                      |
% |  * building           |             phone *  |
% |  * street             |           website *  |
% |  * city, ST ZIP       |             email *  |
% |----------------------------------------------|
% _____________________ HRULE ___________________

\hspace*{-\parindent}%
\begin{center}
  \vspace{-2em}
  {\LARGE\scshape \myname} \\
\end{center}
\begin{minipage}[t]{.6\linewidth}
\address
\end{minipage}
\hspace*{-\parindent}%
\begin{minipage}[t]{.43\linewidth}
\begin{flushright}
  \textit{tel}: \phone \\
  \textit{web}: \href{\websiteurl}{\websitename} \\
  \textit{twitter}: \href{\twitterurl}{\twitterhandle} \\
  \textit{email}: \href{mailto:\email}{\email}
\end{flushright}
\end{minipage}
\begin{center}
  \vspace{-.5em}
  \rule{\textwidth}{1pt}  
\end{center}

% ==============================================================================
% CV BODY: CHANGE THINGS STARTING HERE
% ==============================================================================

% ======================================
% APPOINTMENTS
% ======================================

% Uncomment to add if you have a current appointment.

% \section*{Academic Appointment}
% \begin{tabularx}{\linewidth}{\twocols}
%   2018 - present & Assistant Professor of University, University of
%   University City \\
% \end{tabularx}

% ======================================
% EDUCATION
% ======================================

% Use tabularx (for flexible spacing) to align nicely

\section*{Education}
\begin{tabularx}{\linewidth}{\threecols}
  2023 (expected) & Statistics, University of Michigan & Ph.D. \\
  2018 & Statistics, University of British Columbia & M.Sc. \\
  2016 & Actuarial Science, Simon Fraser University & B.Sc. \\
\end{tabularx}

% ======================================
% RESEARCH INTERESTS
% ======================================

% This will make a horizontal list of items separated by semi-colons. 

\section*{Research Interests}

\hspace{-.25em}\begin{itemize*}[itemjoin={{; }}, label={}]
\item High-dimensional inference
\item Graphical models 
\item (approximate) Bayesian inference
\item Generative models for spatio-temporal processes
\item Applications in natural sciences
\end{itemize*}

% ======================================
% PUBLICATIONS
% ======================================

% Use the same keyword that you've placed in your bib file entry to
% filter only those references you want to place here.

\begin{refsection}
\section*{Publications}
\nocite{*}                          % cite everything
\printbibliography[heading = none,  % no heading (e.g., "References")
keyword = article,                  % FILTER BY THIS KEYWORD
check = recent,                     % check year filter from above
env = mybib]                        % use mybib style
\end{refsection}

% ======================================
% SUBMITTED PAPERS
% ======================================

\begin{refsection}
\section*{Submitted Papers}
\nocite{*}
\printbibliography[heading = none,
keyword = submitted,
check = recent,
env = mybib]
\end{refsection}

% ======================================
% WORKING PAPERS
% ======================================

% \begin{refsection}
% \section*{Working Papers}
% \nocite{*}
% \printbibliography[heading = none,
% keyword = working,
% check = recent,
% env = mybib]
% \end{refsection}

% ======================================
% PRESENTATIONS
% ======================================

% This section differentiates between conference presentations and
% invited presentations. If you only have one or the other, just
% delete the subsections and have one PUBLICATIONS section.

\section*{Presentations}
\subsection*{Contributed presentations}
\begin{refsection}
\nocite{*}
\printbibliography[heading = none,
keyword = conference,               % must have "conference" AND 
keyword = contributed,                     % "peer" to differentiate from "invited"
env = mybib,
check = recent]                     
\end{refsection}                    

\subsection*{Invited presentations}
\begin{refsection}
\nocite{*}
\printbibliography[heading = none,
keyword = conference,               % must have "conference" AND 
keyword = invited,                  % "invited" to differentiate from "peer"
env = mybib,
check = recent]                   
\end{refsection}                   

% ======================================
% TEACHING EXPERIENCE
% ======================================

% Add what makes sense for your field, but if you have links to course
% materials online, be sure to incorporate them below. (NB: the
% backslash on the underscore just b/c TeX treats that differently)

\section*{Teaching}

% \subsection*{Instructor}
% \subsubsection*{University $\mid$ Department}
% \begin{tabularx}{\linewidth}{\twocols}
% YYYY&COURSE NUM: \textit{Course name} \\
% &\hspace{1em}One sentence description (optional)  \\
% &\hspace{1em}{\itshape Website/Course materials:}
% \href{https://github.com/btskinner/tex\_cv}{github.com/btskinner/tex\_cv} \\
% \end{tabularx}

\subsection*{Teaching Assistant}
\subsubsection*{University of Michigan $\mid$ Department of Statistics}
\begin{tabularx}{\linewidth}{\twocols}
2020 & STATS 551: \textit{Bayesian Modelling and Computation} \\
&\hspace{1em}{\itshape Description:} Graduate-level introduction to Bayesian inference.  \\
&\hspace{1em}{\itshape Course materials:} \href{https://ywa136.github.io/files/STATS551WIN2020Syllabus.pdf}{ywa136.github.io/teaching/stats551\_2020Winter} \\

2019 & STATS 306: \textit{Introduction to Statistical Computing} \\
&\hspace{1em}{\itshape Description:} Senior undergraduate-level introductory statistical computing course based on the R programming language. \\
&\hspace{1em}{\itshape Course materials:}
\href{https://github.com/ywa136/stats306_labs}{https://ywa136.github.io/teaching/stats306\_2019Fall} \\

2019 & STATS 426: \textit{Introduction to Theoretical Statistics} \\
&\hspace{1em}{\itshape Description:} Senior undergraduate-level introduction to theoretical statistics.  \\
% &\hspace{1em}{\itshape Website/Course materials:}
% \href{https://ywa136.github.io/files/STATS551WIN2020Syllabus.pdf}{Syllabus} \\

2018 & STATS 250: \textit{Introduction to Statistics and Data Analysis} \\
&\hspace{1em}{\itshape Description:} Junior undergraduate-level course in applied statistical methodology for data analysis.  \\
% &\hspace{1em}{\itshape Website/Course materials:}
% \href{https://ywa136.github.io/files/STATS551WIN2020Syllabus.pdf}{Syllabus} \\
\end{tabularx}

\subsubsection*{University of British Columbia $\mid$ Department of Statistics}
\begin{tabularx}{\linewidth}{\twocols}
2017 & STATS 251: \textit{Introductory Probability and Statistics} \\
&\hspace{1em}{\itshape Description:} Introductory statistics and probability course for junior undergraduates major in statstics.  \\
% &\hspace{1em}{\itshape Website/Course materials:}
% \href{https://github.com/btskinner/tex\_cv}{github.com/btskinner/tex\_cv} \\

2016 & STATS 200: \textit{Elementary Statistics for Applications} \\
&\hspace{1em}{\itshape Description:} Introductory statistics course for junior undergraduates major in arts and social sciences.  \\
\end{tabularx}

% ======================================
% FELLOWSHIPS AND HONORS
% ======================================

% Again, use a table for nicely aliged elements.

\section*{Awards, Fellowships, and Honors}
\begin{tabularx}{\linewidth}{\threecols}
2018 - 2019 & Department of Statistics Fellowship, UMich  \\ % can add \$ AMOUNT
2018 & International Doctoral Fellowship, UBC. (Declined; \textbf{Five year \$$35,000$ per year fellowship awarded to the top $15$ admitted international doctoral students across the university.}) \\
2017 & Travel Award for CBMS Regional Conference on Spatial Statistics. \\
2016 & Statistics \& Actuarial Science Endowment Award: Academic Merit, SFU. (\textbf{Awarded to undergraduate student with the best academic performance across the department in the academic year.}) \\
2015 & VP Research - Undergraduate Student Research Award, SFU. (\textbf{Awarded for the research project ``Robust program for automated statistical imputation of the Alzheimer's Disease Neuroimaging Initiative (ADNI) data''.}) \\
2014 & April Allen Memorial Undergraduate Scholarship, SFU. \\
2014 & President’s Honour List, SFU. \\
2013 - 2016 & Dean’s Honour List, SFU. \\
\end{tabularx}

% ======================================
% PROFESSIONAL MEMBERSHIPS
% ======================================

% This time the list is separated by bullets as an example; use
% whatever seems most aesthetically pleasing to your eyes.

\section*{Services}

\hspace{-.25em}\begin{itemize*}[itemjoin={{ $\bullet$}}, label={}]
\item Reviewer for AISTATS 2021
% \item Organization 2
% \item Organization 3
% \item Organization 4
% \item Organization 5
\end{itemize*}

% ======================================
% OTHER
% ======================================

%%%%%%%%%%%%%%%%%%%%%%%%%%%%%%%%%%%%%%%%%%%%%%%%%%%%%%%%%%%%%%%%%%%%%%%%%%%%%%%%
% END DOCUMENT
%%%%%%%%%%%%%%%%%%%%%%%%%%%%%%%%%%%%%%%%%%%%%%%%%%%%%%%%%%%%%%%%%%%%%%%%%%%%%%%%

\end{document}
